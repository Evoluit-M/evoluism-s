
\documentclass[11pt,a4paper]{article}

% --- Encoding & Fonts
\usepackage[utf8]{inputenc}
\usepackage[T1]{fontenc}

% --- Geometry
\usepackage[margin=1in]{geometry}

% --- Math & Symbols
\usepackage{amsmath,amssymb,mathtools}
\usepackage{siunitx}

% --- Tables
\usepackage{booktabs}
\usepackage{adjustbox}

% --- Graphics
\usepackage{graphicx}
\graphicspath{{./}{figures/}} % ищет и в корне, и в подпапке figures

% --- Hyperlinks
\usepackage{hyperref}
\hypersetup{
  colorlinks=true,
  linkcolor=blue,
  citecolor=blue,
  urlcolor=blue,
  breaklinks=true
}

% --- Floats, Captions, Typography
\usepackage{float}
\usepackage{caption}
\usepackage[section]{placeins}
\usepackage{parskip}
\usepackage{enumitem}
\setlist[itemize]{leftmargin=*, labelsep=0.5em}
\setlist[enumerate]{leftmargin=*, labelsep=0.5em}
\usepackage{xcolor}

% --------------------------------------------------------------
\title{\textbf{Evoluism (S): Cognitive Capital as Order Parameter in the Co-Evolutionary Dynamics of Complex Adaptive Systems}}
\author{Evoluit-M \\ \textit{Independent Theoretical Research Evoluism Initiative}}
\date{08 November 2025}

\begin{document}
\maketitle

% --------------------------------------------------------------
\begin{abstract}
Evoluism (S) introduces \emph{cognitive capital} ($C$) as an \emph{order parameter} governing the co-evolutionary acceleration of complex adaptive systems. Drawing on synergetics and general systems theory, innovation trajectories are modeled as self-organizing processes modulated by positive feedback from $C$. Using a dynamic panel of 85 countries (2001--2024, $n \approx 2{,}040$), a semi-log system-dynamics estimator identifies a robust coordination coefficient $\beta_{\text{synergy}} = 0.70$ ($p < 0.01$), indicating that a one–standard-deviation increase in $C$ amplifies expected patent intensity by $\approx +101\%$. A logistic differential equation serves as a mechanistic archetype: $\frac{dX}{dt} = \alpha X (1 - X/K) + \beta_{\text{synergy}} C$, where $C$ elevates the system’s carrying capacity and growth rate. Cross-scale validation is planned for 2026 across biological (Drosophila), computational (RL agents), and epigenetic (mice) systems. All data, code, and replication environments are open source.

\textbf{Continuity note.} This paper extends the scientific line of research within the Evoluism framework and complements the conceptual foundation published as \textit{“EVOLUISM: Three Reflections of One Reality”} (Zenodo: \href{https://doi.org/10.5281/zenodo.17547103}{10.5281/zenodo.17547103}). Conceptual aspects are referenced for context only; this work is strictly empirical and methodological.
\end{abstract}

\section*{Keywords}
Complex adaptive systems; co-evolution; order parameter; positive feedback; synergetics; self-organization; reproducibility; falsifiability

% --------------------------------------------------------------
\section{Introduction}

Complex systems—economies, ecosystems, or cultures—evolve not through linear causation but via \emph{co-evolutionary feedback} between components operating at multiple scales \cite{haken1977,bertalanffy1968}. In such systems, innovation is not merely an output but a \emph{collective variable} reflecting systemic coordination. Evoluism (S) proposes that this coordination is modulated by an emergent property: \textbf{cognitive capital} ($C$), defined as the capacity of a system to generate, store, and deploy structured information for adaptive transformation.

Far from a static stock, $C$ functions as a coordination or \emph{feedback gain} in the sense of synergetics \cite{haken1977}: it constrains subsystem dynamics, reduces effective degrees of freedom, and drives the system toward higher-level attractors. In national innovation systems, $C$ integrates:
\begin{itemize}
  \item \emph{Tertiary education} — memory and skill transmission,
  \item \emph{R\&D intensity} — exploratory variation,
  \item \emph{Institutional freedom} — selective retention and recombination.
\end{itemize}
These are not independent inputs but coherently aligned under $C$.

$C$ is operationalized as the first principal component (PC1, 78\% variance explained) of the standardized triad above. A semi-log dynamic panel (Arellano--Bond GMM) serves not as a causal claim but as a phase-space mapping of co-evolutionary trajectories:
\begin{equation}
\label{eq:semilog}
\log(X_{it} + 1) = \rho \log(X_{i,t-1} + 1) + \beta_{\text{synergy}} C_{it} + \beta_2 \log(\text{GDP}_{it}) + \alpha_i + \gamma_t + \varepsilon_{it},
\end{equation}
where $X_{it}$ is patent intensity, $\rho$ reflects inertial self-organization, and $\beta_{\text{synergy}} = 0.70$ quantifies the average coordination gain per standard deviation of $C$.

Complementing this, a minimal archetypal model captures bounded growth with cognitive modulation:
\begin{equation}
\label{eq:logistic}
\frac{dX}{dt} = \alpha X \left(1 - \frac{X}{K}\right) + \beta_{\text{synergy}} C.
\end{equation}
Here, $C$ acts as a control parameter that shifts equilibrium and accelerates convergence—analogous to temperature in a laser or stress in a buckling beam.

% --------------------------------------------------------------
\section{Theoretical Framework}

Equation \eqref{eq:semilog} maps the joint evolution of innovation ($X_{it}$) and cognitive capital ($C_{it}$) in the presence of inertial ($\rho$) and environmental ($\log(\text{GDP}_{it})$) constraints. Estimation uses Arellano--Bond GMM (two-step robust, Windmeijer correction, collapsed instruments, lag depth 2--3; $N_{\text{instr}} = 68$, $N_{\text{instr}}/N = 68/85$).  
A System GMM variant (Blundell--Bond) yields similar coefficients, confirming robustness to Nickell bias.

Equation \eqref{eq:logistic} serves as a qualitative archetype of bounded self-organization under cognitive forcing. $C$ modulates both intrinsic growth rate and effective carrying capacity $K(C)$.

% --------------------------------------------------------------
\section{Data and Construction of the Order Parameter $C$}

Dataset: 85 countries, 2001--2024 ($n \approx 2{,}040$).

\textbf{Dependent variable:} $X_{it}$ = patents per million population (WIPO/OECD).  
\textbf{Controls:} $\log(\text{GDP per capita}_{it})$, population, country and time effects.

\textbf{Order parameter $C$:} first principal component (PC1) of z-scored inputs:
\[
C_{it} = \ell_E Z(\text{Education}_{it}) + \ell_R Z(\text{R\&D}_{it}) + \ell_F Z(\text{Freedom}_{it}).
\]
PC1 explains 78\% of total variance.

\begin{table}[H]
\centering
\begin{adjustbox}{max width=\textwidth}
\begin{tabular}{@{}lccc@{}}
\toprule
Component & Education (years) & R\&D (\% GDP) & Economic Freedom \\
\midrule
PC1 loading $\ell$ & 0.48 & 0.31 & 0.21 \\
\bottomrule
\end{tabular}
\end{adjustbox}
\caption{Loadings of variables composing the order parameter $C$.}
\end{table}

One additional year of tertiary education shifts $C$ by $\kappa \approx 0.32$ SD (see Appendix~A).

% --------------------------------------------------------------
\section{Empirical Results}

\subsection{Cross-sectional Snapshot (2024)}
\[
Y_i \sim \text{NB}(\mu_i, \theta), \quad \log \mu_i = \beta_0 + \beta_{\text{synergy}} C_i + \beta_2 \log(\text{GDP}_i) + \log(\text{pop}_i).
\]
Result: $\beta_{\text{synergy}} = 0.92$ ($p < 0.01$), pseudo-$R^2 = 0.41$.

\subsection{Dynamic Panel Mapping (2001--2024)}
\[
\rho \approx 0.63, \quad \beta_{\text{synergy}} = 0.70 \ (p < 0.01),
\]
Hansen $p = 0.31$, AR(2) $p = 0.22$. Robust across specifications.

\subsection{Visual Baseline (Reproduced from v2)}
\begin{figure}[H]
  \centering
  \includegraphics[width=\linewidth]{Figure_S1_Patents_vs_Education_2021.png}
  \caption{\textbf{Figure S1.} Patents per million vs.\ education (2021). Color gradient encodes institutional index. Reproduced from version v2 to preserve the original visual baseline used for macro-level validation.}
  \label{fig:s1}
\end{figure}
\FloatBarrier

\subsection{Robustness}
$\beta_{\text{synergy}} \in [0.65, 0.78]$ across fixed/random effects, leave-one-out, and PCA windows. $\Delta$AIC = $-142$ favors the full model (Appendix~B).

% --------------------------------------------------------------
\section{Discussion}

The coordination gain $\beta_{\text{synergy}} = 0.70$ implies that a one–standard-deviation increase in $C$ yields:
\[
100\!\left(e^{\beta_{\text{synergy}}} - 1\right) \approx +101\%
\]
in expected patent intensity. One extra year of tertiary education shifts $C$ by $\kappa \approx 0.32$ SD:
\[
100\!\left(e^{\beta_{\text{synergy}}\kappa} - 1\right) \approx +25\%.
\]
Interpretation holds for log-transformed $X$; results in low-patent regimes may overestimate marginal effects.  
Economically, this upper-bound effect parallels Romer’s (1990) human capital externalities but is derived empirically as a coordination feedback rather than an exogenous growth factor.

\subsection{Limitations}
\begin{itemize}
\item $C$ correlates with innovation through shared inputs; future work will test instrumented variants (e.g., historical education exposure as IV).
\item PCA collinearity and GDP overlap require further control (VIF diagnostics planned).
\item Macro-aggregation obscures micro-dynamics; cross-scale experiments (2026) will test micro-level analogues.
\end{itemize}

% --------------------------------------------------------------
\section{Cross-Scale Validation Protocols (2026)}

\begin{table}[H]
\centering
\begin{tabular}{@{}lll@{}}
\toprule
System & Intervention & Expected Effect \\
\midrule
\textit{Drosophila} & Enriched learning environment & +15--20\% fitness \\
AI (RL agents) & Extended pre-training & +30--35\% efficiency \\
Mice (epigenetics) & Maternal stress & Intergenerational change ($p < 0.05$) \\
\bottomrule
\end{tabular}
\caption{Planned cross-scale validation of $C$-driven coordination (Q1 2026).}
\end{table}

% --------------------------------------------------------------
\section{Falsifiability}

\begin{table}[H]
\centering
\begin{tabular}{@{}ll@{}}
\toprule
Prediction & Falsification Condition \\
\midrule
$\beta_{\text{synergy}} > 0$ & $\beta_{\text{synergy}} \le 0$ post-diagnostics \\
1 SD $C \mapsto +101\%$ $X$ & No amplification across specifications \\
Per-year education $\mapsto +25\%$ & No $\Delta X$ at $\kappa \approx 0.32$ \\
Micro-level $\Delta F > 0$ & No gain (power $\ge 0.9$) \\
\bottomrule
\end{tabular}
\caption{Core falsifiability criteria.}
\end{table}

% --------------------------------------------------------------
\section{Reproducibility}

All data (OECD, UNESCO, WIPO, World Bank), code (Stata/R), and Docker environment are available at:\\
\url{https://github.com/Evoluit-M/evoluism-s}

% --------------------------------------------------------------
\section{Conclusions}

Evoluism (S) establishes $C$ as a measurable coordination parameter in co-evolutionary systems. The feedback gain $\beta_{\text{synergy}} = 0.70$ is robust ($p < 0.01$).  
Future validation will test feedback coherence across levels of organization.

% --------------------------------------------------------------
\section{Future Work}
\begin{enumerate}
  \item Nonlinear $C \cdot X$ interactions and threshold bifurcations.
  \item IV and placebo PCA to address circularity.
  \item Spatial and Driscoll--Kraay SE extensions.
  \item Integration of macro and micro validation programs.
\end{enumerate}

% --------------------------------------------------------------
\section*{Appendix A: Calibration of $\kappa$}
\[
\kappa = \frac{\ell_E}{\sqrt{\lambda_1}} \cdot \frac{1}{\sigma_{\text{years}}}, \quad \kappa \approx 0.32.
\]

\section*{Appendix B: Model Selection}
\begin{table}[H]
\centering
\begin{tabular}{@{}lcc@{}}
\toprule
Model & AIC & $\Delta$AIC \\
\midrule
Baseline (no $C$) & 4120 & -- \\
Full (with $C$) & 3978 & -142 \\
\bottomrule
\end{tabular}
\end{table}

% --------------------------------------------------------------
\section*{Acknowledgments}
Thanks to OECD, UNESCO, WIPO, World Bank, Heritage Foundation, and the open-science community.

\vspace{1cm}
\begin{center}
\small
© 2025 Evoluit-M. Licensed under CC BY 4.0.\\
Zenodo DOI: \href{https://doi.org/10.5281/zenodo.17565066}{10.5281/zenodo.17565066}\\
Conceptual foundation: \href{https://doi.org/10.5281/zenodo.17547103}{10.5281/zenodo.17547103}\\
Previous empirical version: \href{https://doi.org/10.5281/zenodo.17454336}{10.5281/zenodo.17454336}\\
Data \& Code Repository: \href{https://github.com/Evoluit-M/evoluism-s}{https://github.com/Evoluit-M/evoluism-s}
\end{center}


% --------------------------------------------------------------
\begin{thebibliography}{9}
\bibitem{haken1977} Haken, H. \textit{Synergetics: An Introduction}. Springer, 1977.
\bibitem{bertalanffy1968} von Bertalanffy, L. \textit{General System Theory}. George Braziller, 1968.
\bibitem{romer1990} Romer, P. M. ``Endogenous Technological Change.'' \textit{Journal of Political Economy}, 1990.
\bibitem{boyd2005} Boyd, R., \& Richerson, P. J. \textit{The Origin and Evolution of Cultures}. Oxford University Press, 2005.
\bibitem{arellano1991} Arellano, M., \& Bond, S. ``Some Tests of Specification for Panel Data.'' \textit{Review of Economic Studies}, 1991.
\end{thebibliography}

\end{document}